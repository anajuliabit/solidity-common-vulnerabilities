% Created 2023-10-09 Mon 17:49
% Intended LaTeX compiler: pdflatex
\documentclass[12pt]{article}
\usepackage[utf8]{inputenc}
\usepackage[T1]{fontenc}
\usepackage{graphicx}
\usepackage{longtable}
\usepackage{wrapfig}
\usepackage{rotating}
\usepackage[normalem]{ulem}
\usepackage{amsmath}
\usepackage{amssymb}
\usepackage{capt-of}
\usepackage{hyperref}
\usepackage[utf8]{inputenc}
\usepackage{sbc-template}
\usepackage{graphicx,url}
\address{Universidade do Sul de Santa Catarina (UNISUL)\\ Tubarão - SC - Brasil\\ anajuliabit@gmail.com}
\sloppy
\usepackage{biblatex}

\addbibresource{references.bib}
\author{Ana Julia Bittencourt Fogaça}
\date{\today}
\title{Explorando e classificando bugs comumente encontrados em contratos inteligentes}
\hypersetup{
 pdfauthor={Ana Julia Bittencourt Fogaça},
 pdftitle={Explorando e classificando bugs comumente encontrados em contratos inteligentes},
 pdfkeywords={},
 pdfsubject={},
 pdfcreator={Emacs 29.1 (Org mode 9.7)}, 
 pdflang={Portuges}}
\begin{document}

\maketitle

\section{Abstract}
\label{abstract}
\section{Resumo}
\label{resumo}
\section{Introdução}
\label{sec:orgc5c846d}
A tecnologia blockchain, primeiramente introduzida por Satoshi Nakamoto em 2008,
é identificada como uma megatendência computacional capaz de revolucionar
múltiplos setores industriais\autocite{TechnologyTippingPoints}. As características distintas de
segurança, transparência e rastreabilidade inerentes à blockchain têm
incentivado uma ampla gama de setores a explorar seu uso na reestruturação de
suas operações fundamentais. A aplicabilidade dessa tecnologia ultrapassa o
domínio das criptomoedas, abarcando setores como pagamentos, gerenciamento de
identidade, saúde, eleições governamentais e outros\autocite{BlockchainAdoptionsMaritime}.

A publicação do whitepaper do Ethereum em 2014 simbolizou um avanço considerável
 evolução da tecnologia blockchain\autocite{EthereumWhitepaper}. Diferentemente do Bitcoin,
concebido originalmente como uma moeda digital, o Ethereum inaugurou uma
funcionalidade disruptiva no campo da tecnologia blockchain: os contratos
inteligentes. A inovação trazida pelo Ethereum reside na incorporação de uma
máquina virtual capaz de processar códigos em linguagens de programação
\textit{Turing complete} na blockchain, habilitando assim a construção de
aplicativos descentralizados. Devido as características inerentes a tecnologia blockchain,
como o fato de seu código ser aberto e qualquer pessoa pode interagir com os
contratos inteligentes - descentralização, os aplicativos que rodam no Ethereum
são sucetíveis a vulnerabilidades que podem ser exploradas por hackers,
resultando em grande prejuízo financeiro para os protocolos e usuários dos
mesmos. Apenas no primeiro trimestre de 2023, 320 milhões de dólares foram
perdidos devido a ataque de hackers no Ethereum\autocite{HereHowMuch}. Uma maneira
de combater a ação de hackers, é através de incentivos financeiros. Procurando
proteger seus usuários, protocolos descentralizados costumam oferecer "Bug
Bounties", que são concursos oferecendo recurso financeiro em troca de
vulnerabilidades encontradas por "hackers do bem". Devido a demanda crescente
pela tecnologia de contrato inteligentes nos últimos anos a projeção de
crescimento anual de 2023 a 2030 é de 82.2\%\autocite{SmartContractsMarket}, o
presente artigo tem como objetivo identificar os bugs comumente encontrados nas
diferentes categorias de contratos inteligentes e classificá-los, identificando
possíveis dificuldades na identificação dos mesmos. Para isso, foi feito um
estudo com base em competições realizadas entre janeiro a setembro de 2023
retiradas de diferentes plataformas de Bug Bounties.
\section{Revisão bibliográfica}
\label{sec:org3ab884d}

O que é EVM, EOA, contracts, transactions (nonce).
\section{Metodologia}
\label{sec:org02b35ec}
\subsection{Perguntas}
\label{sec:orgaaa1a24}
\begin{itemize}
\item Categorizando bugs
\end{itemize}
\subsection{Categorias dos protocolos}
\label{sec:org0b6e749}
\begin{itemize}
\item Liquid Staking: Protocols that enable you to earn staking rewards on your tokens while also providing a tradeable and liquid receipt for your staked position
\item Lending: Protocols that allow users to borrow and lend assets
\item Dexes: Protocols where you can swap/trade cryptocurrency
\item Bridge: Protocols that bridge tokens from one network to another
\item CDP: Protocols that mint its own stablecoin using collateralized lending
\item Services: Protocols that provide a service to the user
\item Yield: Protocols that pay you a reward for your staking/LP on their platform
\item RWA: Protocols that involve Real World Assets, such as house tokenization
\item Derivatives: Protocols for betting with leverage
\item Yield Aggregator: Protocols that aggregated yield from diverse protocols
\item Cross Chain: Protocols that add interoperability between different blockchains
\item Synthetics: Protocol that created a tokenized derivative that mimics the value of another asset.
\item Launchpad: Protocols that launch new projects and coins
\item Indexes: Protocols that have a way to track/created the performance of a group of related assets
\item Liquidity manager: Protocols that manage Liquidity Positions in concentrated liquidity AMMs
\item Insurance: Protocols that are designed to provide monetary protections
\item Privacy: Protocols that have the intention of hiding information about transactions
\item Infrastructure
\item Algo-Stables: Protocols that provide algorithmic coins to stablecoins
\item Payments: Protocols that offer the ability to pay/send/receive cryptocurrency
\item Leveraged Farming: Protocols that allow you to leverage yield farm with borrowed money
\item Staking Pool: Refers to platforms where users stake their assets on native blockchains to help secure the network and earn rewards. Unlike Liquid Staking, users don't receive a token representing their staked assets, and their funds are locked up during the staking period, limiting participation in other DeFi activities
\item NFT Marketplace: Protocols where users can buy/sell/rent NFTs
\item NFT Lending: Protocols that allow you to collateralize your NFT for a loan
\item Options: Protocols that give you the right to buy an asset at a fixed price
\item Options Vault: Protocols that allow you to deposit collateral into an options strategy
\item Prediction Market: Protocols that allow you to wager/bet/buy in future results
\item Decentralized Stablecoin: Coins pegged to USD through decentralized mechanisms
\item Farm: Protocols that allow users to lock money in exchange for a protocol token
\item Uncollateralized Lending:Protocol that allows you to lend against known parties that can borrow without collaterall
\item Reserve Currency: OHM forks: Protocols that uses a reserve of valuable assets acquired through bonding and staking to issue and back its native token
\item RWA Lending: Protocols that bridge traditional finance and blockchain ecosystems by tokenizing real-world assets for use as collateral or credit assessment, enabling decentralized lending and borrowing opportunities.
\item Gaming: Protocols that have gaming components
\item Oracle: Protocols that connect data from the outside world (off-chain) with the blockchain world (on-chain)
\item P2P File distributoin system
\item DAO: A decentralized autonomous organization (DAO) is an emerging form of legal structure that has no central governing body and whose members share a common goal to act in the best interest of the entity. Popularized through cryptocurrency enthusiasts and blockchain technology, DAOs are used to make decisions in a bottom-up management approach.
\end{itemize}

Fonte: \url{https://defillama.com/categories}
\subsection{Classificação dos bugs}
\label{sec:orgde316f7}
\begin{itemize}
\item O1: We cannot access the source code of the project.
\item O2: Bugs that occur in off-chain components
\item O3: Smart contracts are written in another language
\item C1: Mempool Manipulation / Front-Running Vulnerabilities, (e.g sandwich attacks, flash-loan exploits)
\item C3: Erroneous state updates.
\begin{itemize}
\item C3-1: Missing state update.
\item C3-2: Incorrect state updates, e.g., a state update that should not be there.
\end{itemize}
\item C5: Privilege escalation and access control issues.
\begin{itemize}
\item C5-1: Users can update privileged state variables arbitrarily (caused by lack of ID-unrelated input sanitization).
\item C5-2: Users can invoke some functions at a time they should not be able to do so.
\item C5-3: Privileged functions can be called by anyone or at any time.
\item C5-4: User funds can get locked due to missing/wrong withdraw code
\item C5-6: Privileged users can profit unfarly
\end{itemize}
\item C6: Wrong Math / Erroneous accounting.
Wrong Math refers to a potential issue where mathematical operations within a smart contract are implemented incorrectly, leading to inaccurate calculations.
\begin{itemize}
\item C6-1: Incorrect calculating order.
\item C6-2: Returning an unexpected value that deviates from the expected semantics specified for the contract.
\item C6-3: Calculations performed with incorrect numbers (e.g., x = a + b ==> x = a + c, incorrect precisions).
\item C6-4: Other accounting errors (e.g., x = a + b ==> x = a - b).
\end{itemize}
\item C7: Broken business logic
Logic vulnerabilities involve flaws in the business logic or protocols of a smart contract, where the implementation matches the developer's intention, but the underlying logic is inherently flawed.
\begin{itemize}
\item C7-1: Unexpected or missing function invocation sequences (e.g., external calls to dependent contracts,  exploitable sequences leading to malicious fund reallocation or manipulation).
\item C7-2: Unexpected environment or contract conditions (e.g., ChainLink returning outdated data or significant slippage occurring).
\item C7-3: A given function is invoked multiple times unexpectedly.
\item C7-4: Unexpected function arguments.
\end{itemize}
\item C8: Contract implementation-specific bugs. These bugs are difficult to categorize into the above categories.
\item C9: Lack of signature replay protection, e.g missing nonce, hash collision
\item C10: Missing check.
Missing Check refers to a critical oversight in a smart contract's code where a necessary condition or validation is not properly implemented.
\item C11: lack of segregation between users funds
\item C12: Data validation
Data validation vulnerabilities arise when a smart contract does not adequately verify or sanitize inputs, especially those from untrusted sources. This lack of validation can lead to unintended and potentially harmful consequences within the contract’s operations.
\item C13: Whitelist/Blacklist Match
Whitelist/Blacklist Match refers to a potential vulnerability where a smart contract improperly handles addresses based on predefined lists.
\item C14: Arrays
Vulnerabilities related to arrays can arise when developers do not properly handle array indices or fail to validate user inputs.
 would typically be reserved for vulnerabilities that directly arise from mishandling or misinterpreting arrays in the code. For example, if there were out-of-bound reads/writes, deletion mishaps, or issues with array resizing
\item C15: DoS: Denial of Service (DoS) vulnerabilities occur when an attacker can exploit a contract in a way that makes it unresponsive or significantly less efficient. This category includes cases that are not well described by another class and where the primary consequence is contract shut-down or operational inefficiency.
\item C16: Grielf Attack: A gas griefing attack happens when a user sends the amount of gas required to execute the target smart contract, but not its sub calls. In most cases, this results in uncontrolled behavior that could have a dangerous impact on the business logic.
\item C17: Reentry attack - Reentrancy vulnerabilities happen when external contract calls are made before internal state updates, allowing an adversary to recursively call back into the contract, exploiting the inconsistent state.
\item C18: Hardcoded Setting - refers to the practice of embedding fixed values or parameters directly into the source code of a smart contract. This can pose a security risk if the setting needs to be dynamic or adaptable.
\end{itemize}
\subsection{Dados coletados}
\label{sec:org980e649}
Foi feito a curadoria de X bugs classificados com severidade alta
\subsection{Desenvolvimento}
\label{sec:org7698f9a}
\subsection{Categorias}
\label{sec:org458a22e}
\subsection{Dificuldade}
\label{sec:org43bed32}
\section{Referências}
\label{sec:org8d89d1d}
\printbibliography
\end{document}