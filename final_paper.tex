% Created 2023-10-06 Fri 18:08
% Intended LaTeX compiler: pdflatex
\documentclass[11pt]{article}
\usepackage[utf8]{inputenc}
\usepackage[T1]{fontenc}
\usepackage{graphicx}
\usepackage{longtable}
\usepackage{wrapfig}
\usepackage{rotating}
\usepackage[normalem]{ulem}
\usepackage{amsmath}
\usepackage{amssymb}
\usepackage{capt-of}
\usepackage{hyperref}
\author{Ana Julia}
\date{\today}
\title{Explorando e classificando bugs comumente encontrados em contratos inteligentes}
\hypersetup{
 pdfauthor={Ana Julia},
 pdftitle={Explorando e classificando bugs comumente encontrados em contratos inteligentes},
 pdfkeywords={},
 pdfsubject={},
 pdfcreator={Emacs 29.1 (Org mode 9.7)}, 
 pdflang={English}}
\usepackage{biblatex}

\begin{document}

\maketitle
\tableofcontents

\section{Introdução}
\label{sec:org499ae9e}
A tecnologia blockchain, primeiramente introduzida por Satoshi Nakamoto em 2008,
é identificada como uma megatendência computacional capaz de revolucionar
múltiplos setores industriais\cite{1}. As características distintas de
segurança, transparência e rastreabilidade inerentes à blockchain têm
incentivado uma ampla gama de setores a explorar seu uso na reestruturação de
suas operações fundamentais. A aplicabilidade dessa tecnologia ultrapassa o
domínio das criptomoedas, abarcando setores como pagamentos, gerenciamento de
identidade, saúde, eleições governamentais e outros\cite{2}.

A publicação do whitepaper do Ethereum em 2014 simbolizou um avanço considerável
na evolução da tecnologia blockchain\cite{3}. Diferentemente do Bitcoin,
concebido originalmente como uma moeda digital, o Ethereum inaugurou uma
funcionalidade disruptiva no campo da tecnologia blockchain: os contratos
inteligentes. A inovação trazida pelo Ethereum reside na incorporação de uma
máquina virtual capaz de processar códigos em linguagens de programação
\textit{Turing complete} na blockchain, habilitando assim a construção de
aplicativos descentralizados. Estes aplicativos propõem a substituição dos
sistemas de back-end por contratos inteligentes que operam em uma
blockchain\cite{7}. Devido as características inerentes a tecnologia blockchain,
como o fato de seu código ser aberto e qualquer pessoa pode interagir com os
contratos inteligentes - descentralização, os aplicativos que rodam no Ethereum
são sucetíveis a vulnerabilidades que podem ser exploradas por hackers,
resultando em grande prejuízo financeiro para os protocolos e usuários dos
mesmos. Apenas no primeiro trimestre de 2023, 320 milhões de dólares foram
perdidos devido a ataque de hackers no Ethereum\autocite{HereHowMuch}. Uma maneira
de combater a ação de hackers, é através de incentivos financeiros. Procurando
proteger seus usuários, protocolos descentralizados costumam oferecer ``Bug
Bounties'', que são concursos oferecendo recurso financeiro em troca de
vulnerabilidades encontradas por ``hackers do bem''. Devido a demanda crescente
pela tecnologia de contrato inteligentes nos últimos anos a projeção de
crescimento anual de 2023 a 2030 é de 82.2\%\autocite{SmartContractsMarket}, o
presente artigo tem como objetivo identificar os bugs comumente encontrados nas
diferentes categorias de contratos inteligentes e classificá-los, identificando
possíveis dificuldades na identificação dos mesmos. Para isso, foi feito um
estudo com base em competições realizadas entre janeiro a setembro de 2023
retiradas de diferentes plataformas de Bug Bounties.
\section{Revisão bibliográfica}
\label{sec:orgb1bda03}

O que é EVM, EOA, contracts, transactions (nonce).
\section{Metodologia}
\label{sec:orgef954a5}
\subsection{Perguntas}
\label{sec:org317278f}
\begin{itemize}
\item Categorizando bugs
\end{itemize}
\subsection{Categorias dos protocolos}
\label{sec:orgdf936b0}
\begin{itemize}
\item Liquid Staking: Protocols that enable you to earn staking rewards on your tokens while also providing a tradeable and liquid receipt for your staked position
\item Lending: Protocols that allow users to borrow and lend assets
\item Dexes: Protocols where you can swap/trade cryptocurrency
\item Bridge: Protocols that bridge tokens from one network to another
\item CDP: Protocols that mint its own stablecoin using collateralized lending
\item Services: Protocols that provide a service to the user
\item Yield: Protocols that pay you a reward for your staking/LP on their platform
\item RWA: Protocols that involve Real World Assets, such as house tokenization
\item Derivatives: Protocols for betting with leverage
\item Yield Aggregator: Protocols that aggregated yield from diverse protocols
\item Cross Chain: Protocols that add interoperability between different blockchains
\item Synthetics: Protocol that created a tokenized derivative that mimics the value of another asset.
\item Launchpad: Protocols that launch new projects and coins
\item Indexes: Protocols that have a way to track/created the performance of a group of related assets
\item Liquidity manager: Protocols that manage Liquidity Positions in concentrated liquidity AMMs
\item Insurance: Protocols that are designed to provide monetary protections
\item Privacy: Protocols that have the intention of hiding information about transactions
\item Infrastructure
\item Algo-Stables: Protocols that provide algorithmic coins to stablecoins
\item Payments: Protocols that offer the ability to pay/send/receive cryptocurrency
\item Leveraged Farming: Protocols that allow you to leverage yield farm with borrowed money
\item Staking Pool: Refers to platforms where users stake their assets on native blockchains to help secure the network and earn rewards. Unlike Liquid Staking, users don't receive a token representing their staked assets, and their funds are locked up during the staking period, limiting participation in other DeFi activities
\item NFT Marketplace: Protocols where users can buy/sell/rent NFTs
\item NFT Lending: Protocols that allow you to collateralize your NFT for a loan
\item Options: Protocols that give you the right to buy an asset at a fixed price
\item Options Vault: Protocols that allow you to deposit collateral into an options strategy
\item Prediction Market: Protocols that allow you to wager/bet/buy in future results
\item Decentralized Stablecoin: Coins pegged to USD through decentralized mechanisms
\item Farm: Protocols that allow users to lock money in exchange for a protocol token
\item Uncollateralized Lending:Protocol that allows you to lend against known parties that can borrow without collaterall
\item Reserve Currency: OHM forks: Protocols that uses a reserve of valuable assets acquired through bonding and staking to issue and back its native token
\item RWA Lending: Protocols that bridge traditional finance and blockchain ecosystems by tokenizing real-world assets for use as collateral or credit assessment, enabling decentralized lending and borrowing opportunities.
\item Gaming: Protocols that have gaming components
\item Oracle: Protocols that connect data from the outside world (off-chain) with the blockchain world (on-chain)
\item P2P File distributoin system
\item DAO: A decentralized autonomous organization (DAO) is an emerging form of legal structure that has no central governing body and whose members share a common goal to act in the best interest of the entity. Popularized through cryptocurrency enthusiasts and blockchain technology, DAOs are used to make decisions in a bottom-up management approach.
\end{itemize}

Fonte: \url{https://defillama.com/categories}
\subsection{Classificação dos bugs}
\label{sec:org399913e}
\begin{itemize}
\item O1: We cannot access the source code of the project.
\item O2: Bugs that occur in off-chain components
\item O3: Smart contracts are written in another language
\item C3: Erroneous state updates.
\begin{itemize}
\item C3-1: Missing state update.
\item C3-2: Incorrect state updates, e.g., a state update that should not be there.
\end{itemize}
\item C5: Privilege escalation and access control issues.
\begin{itemize}
\item C5-1: Users can update privileged state variables arbitrarily (caused by lack of ID-unrelated input sanitization).
\item C5-2: Users can invoke some functions at a time they should not be able to do so.
\item C5-3: Privileged functions can be called by anyone or at any time.
\item C5-4: User funds can get locked due to missing/wrong withdraw code
\item C5-6: Privileged users can profit unfarly
\end{itemize}
\item C6: Erroneous accounting.
\begin{itemize}
\item C6-1: Incorrect calculating order.
\item C6-2: Returning an unexpected value that deviates from the expected semantics specified for the contract.
\item C6-3: Calculations performed with incorrect numbers (e.g., x = a + b ==> x = a + c, incorrect precisions).
\item C6-4: Other accounting errors (e.g., x = a + b ==> x = a - b).
\end{itemize}
\item C7: Broken business logic
\begin{itemize}
\item C7-1: Unexpected or missing function invocation sequences (e.g., external calls to dependent contracts,  exploitable sequences leading to malicious fund reallocation or manipulation).
\item C7-2: Unexpected environment or contract conditions (e.g., ChainLink returning outdated data or significant slippage occurring).
\item C7-3: A given function is invoked multiple times unexpectedly.
\item C7-4: Unexpected function arguments.
\end{itemize}
\item C8: Contract implementation-specific bugs. These bugs are difficult to categorize into the above categories.
\item C9: Lack of signature replay protection, e.g missing nonce, hash collision
\item C10: Missing check.
Missing Check refers to a critical oversight in a smart contract's code where a necessary condition or validation is not properly implemented.
\item C11: lack of segregation between users funds
\item C12: Data validation
Data validation vulnerabilities arise when a smart contract does not adequately verify or sanitize inputs, especially those from untrusted sources. This lack of validation can lead to unintended and potentially harmful consequences within the contract’s operations.
\item C13: Whitelit/Blacklist Match
Whitelist/Blacklist Match refers to a potential vulnerability where a smart contract improperly handles addresses based on predefined lists.
\item C14: Arrays
Array refers to a data structure that holds multiple elements under a single variable name. Vulnerabilities related to arrays can arise when developers do not properly handle array indices or fail to validate user inputs.
\item C15: DoS: Denial of Service (DoS) vulnerabilities occur when an attacker can exploit a contract in a way that makes it unresponsive or significantly less efficient. This category includes cases that are not well described by another class and where the primary consequence is contract shut-down or operational inefficiency.
\item C16: Grielf Attack: A gas griefing attack happens when a user sends the amount of gas required to execute the target smart contract, but not its sub calls. In most cases, this results in uncontrolled behavior that could have a dangerous impact on the business logic.
\end{itemize}
\subsection{Dados coletados}
\label{sec:org46473fd}
Foi feito a curadoria de 470 bugs classificados com severidade alta

\begin{center}
\begin{tabular}{lllrlllr}
Plataforma & Protocolo & Categoria do protocolo & N de auditores & Descrição & Link & Classificação & Rev\\[0pt]
\hline
Sherlock & Perennial V2 & Derivatives & 4 & Oracle request timestamp and pending position timestamp mismatch can make most position updates invalid & \href{https://github.com/sherlock-audit/2023-07-perennial-judging/issues/42}{Github} & C3-2 & \\[0pt]
Sherlock & Perennial V2 & Derivatives & 1 & Invalid oracle versions can cause desync of global and local positions making protocol lose funds and being unable to pay back all users & \href{https://github.com/sherlock-audit/2023-07-perennial-judging/issues/49}{Github} &  & \\[0pt]
Sherlock & Perennial V2 & Derivatives & 4 & Protocol fee from Market.sol is locked & \href{https://github.com/sherlock-audit/2023-07-perennial-judging/issues/52}{Github} & C5-4 & \\[0pt]
Sherlock & Perennial V2 & Derivatives & 3 & PythOracle:if price.expo is less than 0, wrong prices will be recorded & \href{https://github.com/sherlock-audit/2023-07-perennial-judging/issues/56}{Github} & C6-4 & \\[0pt]
Sherlock & Perennial V2 & Derivatives & 4 & Vault.sol: settleing the 0 address will disrupt accounting & \href{https://github.com/sherlock-audit/2023-07-perennial-judging/issues/62}{Github} &  & \\[0pt]
Sherlock & Perennial V2 & Derivatives & 1 & Keepers will suffer significant losses due to miss compensation for L1 rollup fees & \href{https://github.com/sherlock-audit/2023-07-perennial-judging/issues/91}{Github} &  & \\[0pt]
Sherlock & Blueberry & Leverage Farming & 1 & Stable BPT valuation is incorrect and can be exploited to cause protocol insolvency & \href{https://github.com/sherlock-audit/2023-07-blueberry-judging/issues/97}{Github} &  & \\[0pt]
Sherlock & Blueberry & Leverage Farming & 2 & CurveTricryptoOracle incorrectly assumes that WETH is always the last token in the pool which leads to bad LP pricing & \href{https://github.com/sherlock-audit/2023-07-blueberry-judging/issues/98}{Github} & C8 & \\[0pt]
Sherlock & Blueberry & Leverage Farming & 2 & CurveTricryptoOracle\#getPrice contains math error that causes LP to be priced completely wrong & \href{https://github.com/sherlock-audit/2023-07-blueberry-judging/issues/100}{Github} & C6-3 & \\[0pt]
Sherlock & Blueberry & Leverage Farming & 1 & CVX/AURA distribution calculation is incorrect and will lead to loss of rewards at the end of each cliff & \href{https://github.com/sherlock-audit/2023-07-blueberry-judging/issues/100}{Github} &  & \\[0pt]
Sherlock & Blueberry & Leverage Farming & 1 & wrong bToken's exchangeRateStored used for calculate ColleteralValue & \href{https://github.com/sherlock-audit/2023-07-blueberry-judging/issues/117}{Github} &  & \\[0pt]
Code4Arena & Arbitrum Foundation & DAO & 3 & Signatures can be replayed in `castVoteWithReasonAndParamsBySig()` to use up more votes than a user intended & \href{https://code4rena.com/reports/2023-08-arbitrum}{Code4Arena} & C9 & 1\\[0pt]
Code4Arena & PoolTogether & Yield & 1 & Too many rewards are distributed when a draw is closed & \href{https://code4rena.com/reports/2023-08-pooltogether}{Code4Arena} & C6-3 & \\[0pt]
Code4Arena & PoolTogether & Yield & 16 & rngComplete function should only be called by rngAuctionRelayer & \href{https://code4rena.com/reports/2023-08-pooltogether}{Code4Arena} & C5-3 & \\[0pt]
Sherlock & Tokensoft & Launchpad & 24 & ``Votes'' balance can be increased indefinitely in multiple contracts & \href{https://github.com/sherlock-audit/2023-06-tokensoft-judging/issues/41}{Github} & C5-3 & \\[0pt]
Sherlock & Bond Options & Options & 14 & All funds from Teller contract can be drained because a malicious receiver can call reclaim repeatedly & \href{https://github.com/sherlock-audit/2023-06-bond-judging/issues/79}{Github} & C3-1 & \\[0pt]
Sherlock & Bond Options & Options & 4 & All funds can be stolen from FixedStrikeOptionTeller using a token with malicious decimals & \href{https://github.com/sherlock-audit/2023-06-bond-judging/issues/90}{Github} & C7-1 & \\[0pt]
Sherlock & Symmetrical & Derivatives & 2 & liquidatePartyA requires signature which doesn't have nonce, making possible unfair liquidation and loss of funds for all parties & \href{https://github.com/sherlock-audit/2023-08-symmetrical-judging/issues/5}{Github} & C9 & \\[0pt]
Sherlock & Symmetrical & Derivatives & 2 & liquidatePositionsPartyA limits partyB loss to partyB allocated balance, which can lead to inflated partyB balance and loss of funds & \href{https://github.com/sherlock-audit/2023-08-symmetrical-judging/issues/6}{Github} & C3-2 & \\[0pt]
Sherlock & Cooler Update & Lending & 3 & Can steal gOhm by calling Clearinghouse.claimDefaulted on loans not made by Clearinghouse & \href{https://github.com/sherlock-audit/2023-08-cooler-judging/issues/28}{Github} & C10 & \\[0pt]
Sherlock & Cooler Update & Lending & 10 & At claimDefaulted, the lender may not receive the token because the Unclaimed token is not processed & \href{https://github.com/sherlock-audit/2023-08-cooler-judging/issues/119}{Github} & C3-1 & \\[0pt]
Sherlock & Cooler Update & Lending & 2 & Clearinghouse doesn't approve the MINTR to handle tokens in his name, which bricks the entire function. & \href{https://github.com/sherlock-audit/2023-08-cooler-judging/issues/176}{Github} & C7-1 & \\[0pt]
Sherlock & Cooler Update & Lending & 20 & isCoolerCallback can be bypassed & \href{https://github.com/sherlock-audit/2023-08-cooler-judging/issues/176}{Github} & C8 & \\[0pt]
Sherlock & GFX Labs & Dexes & 6 & Lack of segregation between users' assets and collected fees resulting in loss of funds for the users & \href{https://github.com/sherlock-audit/2023-06-gfx-judging/issues/48}{Github} & C11 & \\[0pt]
Sherlock & GFX Labs & Dexes & 4 & Users' funds could be stolen or locked by malicious or rouge owners & \href{https://github.com/sherlock-audit/2023-06-gfx-judging/issues/54}{Github} & C12 & \\[0pt]
Code4Arena & PoolTogether & Yield & 2 & A malicious user can steal other user's deposits from Vault.sol & \href{https://solodit.xyz/issues/h-02-a-malicious-user-can-steal-other-users-deposits-from-vaultsol-code4rena-pooltogether-pooltogether-git}{Solodit} & C6-3 & 1\\[0pt]
Code4Arena & PoolTogether & Yield & 5 & `\textsubscript{amountOut}` is representing assets and shares at the same time in the `liquidate` function & \href{https://solodit.xyz/issues/h-03-\_amountout-is-representing-assets-and-shares-at-the-same-time-in-the-liquidate-function-code4rena-pooltogether-pooltogether-git}{Solodit} &  & \\[0pt]
Code4Arena & PoolTogether & Yield & 39 & `Vault.mintYieldFee` function can be called by anyone to mint `Vault Shares` to any recipient address & \href{https://solodit.xyz/issues/h-04-vaultmintyieldfee-function-can-be-called-by-anyone-to-mint-vault-shares-to-any-recipient-address-code4rena-pooltogether-pooltogether-git}{Solidit} & C7-4 & \\[0pt]
Code4Arena & PoolTogether & Yield & 10 & Delegated amounts can be forcefully removed from anyone in the `TwabController` & \href{https://solodit.xyz/issues/h-05-delegated-amounts-can-be-forcefully-removed-from-anyone-in-the-twabcontroller-code4rena-pooltogether-pooltogether-git}{Solodit} & C7-4 & \\[0pt]
Code4Arena & PoolTogether & Yield & 8 & Resetting delegation will result in user funds being lost forever & \href{https://solodit.xyz/issues/h-06-resetting-delegation-will-result-in-user-funds-being-lost-forever-code4rena-pooltogether-pooltogether-git}{Solodit} & C3-2 & \\[0pt]
Code4Arena & PoolTogether & Yield & 3 & `\textsubscript{requireVaultCollateralized}()` is called at the beginning of the functions `mintYieldFee()` and `liquidate()` & \href{https://solodit.xyz/issues/h-07-\_requirevaultcollateralized-is-called-at-the-beginning-of-the-functions-mintyieldfee-and-liquidate-code4rena-pooltogether-pooltogether-git}{Solodit} & C7-1 & \\[0pt]
Code4Arena & PoolTogether & Yield & 5 & Increasing reserves breaks ProzePool accounting & \href{https://solodit.xyz/issues/h-08-increasing-reserves-breaks-prizepool-accounting-code4rena-pooltogether-pooltogether-git}{Solodit} & C3-1 & 1\\[0pt]
Code4Arena & PoolTogether & Yield & 2 & `Vault` is not compatible with some ERC4626 vaults & \href{https://solodit.xyz/issues/h-09-vault-is-not-compatible-with-some-erc4626-vaults-code4rena-pooltogether-pooltogether-git}{Solodit} & C8 & \\[0pt]
Sherlock & Dinari & RWA & 4 & Bypass the blacklist restriction because the blacklist check is not done when minting or burning & \href{https://github.com/sherlock-audit/2023-06-dinari-judging/issues/64}{Github} & C13 & \\[0pt]
Sherlock & Unstopabble & Dexes & 1 & Wrong accounting of the storage balances results for the protocol to be in debt even when the bad debt is repaid & \href{https://github.com/sherlock-audit/2023-06-unstoppable-judging/issues/68}{Github} & O3 & 1\\[0pt]
Sherlock & Unstopabble & Dexes & 1 & reduce\textsubscript{margin}\textsubscript{by}\textsubscript{amount} in Vault.reduce\textsubscript{position} is wrongly calculated & \href{https://github.com/sherlock-audit/2023-06-unstoppable-judging/issues/85}{Github} & O3 & 1\\[0pt]
Sherlock & Unstopabble & Dexes & 7 & Vault: The attacker can sandwich attack himself on swaps in open\textsubscript{position}, close\textsubscript{position} and reduce\textsubscript{position} to make a bad debt & \href{https://github.com/sherlock-audit/2023-06-unstoppable-judging/issues/140}{Github} & O3 & 1\\[0pt]
Sherlock & Unstopabble & Dexes & 6 & reduce\textsubscript{position} doesn’t update margin mapping correctly & \href{https://github.com/sherlock-audit/2023-06-unstoppable-judging/issues/143}{Github} & O3 & 1\\[0pt]
Sherlock & Unstopabble & Dexes & 3 & Leverage calculation is wrong & \href{https://github.com/sherlock-audit/2023-06-unstoppable-judging/issues/150}{Github} & O3 & 1\\[0pt]
Sherlock & Unstopabble & Dexes & 11 & Vault: \_update\textsubscript{debt} does not accrue interest & \href{https://github.com/sherlock-audit/2023-06-unstoppable-judging/issues/167}{Github} & O3 & 1\\[0pt]
Sherlock & Unstopabble & Dexes & 6 & Adversary manipulate the middle path when calling execute\textsubscript{dca}\textsubscript{order}, resulting user loss, benefiting the attacker & \href{https://github.com/sherlock-audit/2023-06-unstoppable-judging/issues/182}{Github} & O3 & 1\\[0pt]
Sherlock & Unstopabble & Dexes & 2 & Interested calculated is amplified by multiple of 1000 in \_debt\textsubscript{interest}\textsubscript{since}\textsubscript{last}\textsubscript{update} & \href{https://github.com/sherlock-audit/2023-06-unstoppable-judging/issues/191}{Github} & O3 & 1\\[0pt]
Code4Arena & Nouns DAO & DAO & 5 & User can steal tokens by using duplicated ERC20 tokens as parameter in `NounsDAOLogicV1Fork.quit` & \href{https://solodit.xyz/issues/h-01-user-can-steal-tokens-by-using-duplicated-erc20-tokens-as-parameter-in-nounsdaologicv1forkquit-code4rena-nouns-dao-nouns-dao-git}{Solid} & C14 & 1\\[0pt]
Sherlock & Hubble Exchange & Dexes, Derivatives & 11 & ProcessWithdrawals is still DOS-able & \href{https://solodit.xyz/issues/h-1-processwithdrawals-is-still-dos-able-sherlock-none-hubble-exchange-git}{Solodit} & C15 & 1\\[0pt]
Sherlock & Hubble Exchange & Dexes, Derivatives & 11 & Failed withdrawals from VUSD\#processWithdrawals will be lost forever & \href{https://solodit.xyz/issues/h-2-failed-withdrawals-from-vusdprocesswithdrawals-will-be-lost-forever-sherlock-none-hubble-exchange-git}{Solodit} & C5-4 & \\[0pt]
Sherlock & Hubble Exchange & Dexes, Derivatives & 1 & Rogue validators can manipulate funding rates and profit unfairly from liquidations & \href{https://solodit.xyz/issues/h-3-rogue-validators-can-manipulate-funding-rates-and-profit-unfairly-from-liquidations-sherlock-none-hubble-exchange-git}{Solodit} &  & \\[0pt]
Sherlock & Symmetrical & Derivatives & 13 & setSymbolsPrice() can use the priceSig from a long time ago & \href{https://github.com/sherlock-audit/2023-06-symmetrical-judging/issues/113}{Github} & C12 & \\[0pt]
Sherlock & Symmetrical & Derivatives & 1 & liquidatePositionsPartyB can be used by a malicious liquidator to liquidate only select positions which artificially inflates partyA upnl & \href{https://github.com/sherlock-audit/2023-06-symmetrical-judging/issues/160}{Github} & C7-1 & \\[0pt]
Sherlock & Symmetrical & Derivatives & 8 & PartyA and PartyB nonce is not incremented in any of the liquidation functions which can lead to all protocol funds being stolen & \href{https://github.com/sherlock-audit/2023-06-symmetrical-judging/issues/190}{Github} & C9 & 1\\[0pt]
Sherlock & Symmetrical & Derivatives & 2 & LibMuon Signature hash collision & \href{https://github.com/sherlock-audit/2023-06-symmetrical-judging/issues/214}{Github} & C9 & 1\\[0pt]
Sherlock & Symmetrical & Derivatives & 13 & depositAndAllocateForPartyB is broken due to incorrect precision & \href{https://github.com/sherlock-audit/2023-06-symmetrical-judging/issues/222}{Github} & C6-3 & 1\\[0pt]
Sherlock & Symmetrical & Derivatives & 5 & Accounting error in PartyB's pending locked balance led to loss of funds & \href{https://github.com/sherlock-audit/2023-06-symmetrical-judging/issues/226}{Github} & C6-2 & 1\\[0pt]
Sherlock & Symmetrical & Derivatives & 17 & Liquidation can be blocked by incrementing the nonce & \href{https://github.com/sherlock-audit/2023-06-symmetrical-judging/issues/233}{Github} & C6-2 & 1\\[0pt]
 &  &  &  &  &  &  & \\[0pt]
 &  &  &  &  &  &  & \\[0pt]
 &  &  &  &  &  &  & \\[0pt]
\end{tabular}
\end{center}
\subsection{Desenvolvimento}
\label{sec:orgbb5c84c}
\subsection{Categorias}
\label{sec:org7a179ad}

\subsection{Dificuldade}
\label{sec:org94da180}
\end{document}