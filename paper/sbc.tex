% Created 2023-10-14 Sat 09:42
% Intended LaTeX compiler: pdflatex
\documentclass[12pt]{article}
\usepackage[utf8]{inputenc}
\usepackage[T1]{fontenc}
\usepackage{graphicx}
\usepackage{longtable}
\usepackage{wrapfig}
\usepackage{rotating}
\usepackage[normalem]{ulem}
\usepackage{amsmath}
\usepackage{amssymb}
\usepackage{capt-of}
\usepackage{hyperref}
\usepackage[utf8]{inputenc}
\usepackage{sbc-template}
\usepackage{graphicx,url}
\address{Universidade do Sul de Santa Catarina (UNISUL)\\ Tubarão - SC - Brasil\\ anajuliabit@gmail.com}
\sloppy
\usepackage{biblatex}

\addbibresource{references.bib}
\author{Ana Julia Bittencourt Fogaça}
\date{\today}
\title{Explorando e classificando bugs comumente encontrados em contratos inteligentes}
\hypersetup{
 pdfauthor={Ana Julia Bittencourt Fogaça},
 pdftitle={Explorando e classificando bugs comumente encontrados em contratos inteligentes},
 pdfkeywords={},
 pdfsubject={},
 pdfcreator={Emacs 29.1 (Org mode 9.7)}, 
 pdflang={Portuges}}
\begin{document}

\maketitle

\section{Abstract}
\label{abstract}
\section{Resumo}
\label{resumo}
\section{Introdução}
\label{sec:org86c2027}
A tecnologia blockchain, primeiramente introduzida por Satoshi Nakamoto em 2008,
é identificada como uma megatendência computacional capaz de revolucionar
múltiplos setores industriais\autocite{TechnologyTippingPoints}. As características distintas de
segurança, transparência e rastreabilidade inerentes à blockchain têm
incentivado uma ampla gama de setores a explorar seu uso na reestruturação de
suas operações fundamentais. A aplicabilidade dessa tecnologia ultrapassa o
domínio das criptomoedas, abarcando setores como pagamentos, gerenciamento de
identidade, saúde, eleições governamentais e outros\autocite{BlockchainAdoptionsMaritime}.

A publicação do whitepaper do Ethereum em 2014 simbolizou um avanço considerável
 evolução da tecnologia blockchain\autocite{EthereumWhitepaper}. Diferentemente do Bitcoin,
concebido originalmente como uma moeda digital, o Ethereum inaugurou uma
funcionalidade disruptiva no campo da tecnologia blockchain: os contratos
inteligentes. A inovação trazida pelo Ethereum reside na incorporação de uma
máquina virtual capaz de processar códigos em linguagens de programação
\textit{Turing complete} na blockchain, habilitando assim a construção de
aplicativos descentralizados. Devido as características inerentes a tecnologia blockchain,
como o fato de seu código ser aberto e qualquer pessoa pode interagir com os
contratos inteligentes - descentralização, os aplicativos que rodam no Ethereum
são sucetíveis a vulnerabilidades que podem ser exploradas por hackers,
resultando em grande prejuízo financeiro para os protocolos e usuários dos
mesmos. Apenas no primeiro trimestre de 2023, 320 milhões de dólares foram
perdidos devido a ataque de hackers no Ethereum\autocite{HereHowMuch}. Uma maneira
de combater a ação de hackers, é através de incentivos financeiros. Procurando
proteger seus usuários, protocolos descentralizados costumam oferecer "Bug
Bounties", que são concursos oferecendo recurso financeiro em troca de
vulnerabilidades encontradas por "hackers do bem". Devido a demanda crescente
pela tecnologia de contrato inteligentes nos últimos anos a projeção de
crescimento anual de 2023 a 2030 é de 82.2\%\autocite{SmartContractsMarket}, o
presente artigo tem como objetivo identificar os bugs comumente encontrados nas
diferentes categorias de contratos inteligentes e classificá-los, identificando
possíveis dificuldades na identificação dos mesmos. Para isso, foi feito um
estudo com base em competições realizadas entre janeiro a setembro de 2023
retiradas de diferentes plataformas de Bug Bounties.
\section{Revisão bibliográfica}
\label{sec:orge699f0d}

\subsection{Ethereum}
\label{sec:orgf46f7a0}
\subsection{EVM}
\label{sec:orgdad93c5}
\subsection{Smart contracts}
\label{sec:orgbeb557c}
\subsection{Solidity}
\label{sec:org49b36a4}
\subsection{ERCs}
\label{sec:org9d4627d}
\section{Metodologia}
\label{sec:org4b48d1f}
\subsection{Perguntas da pesquisa}
\label{sec:orgaace812}
\begin{itemize}
\item Q1: Que tipo de vulnerabilidade é mais difícil de ser encontrada por auditores?
\item Q2: Que categoria de protocolo apresenta mais presença de bugs?
\item Q3: Os auditores frequentemente perdem tipos específicos de bugs que são posteriormente explorados?
\item Q4: Qual é o impacto financeiro médio de diferentes tipos de vulnerabilidades?
\item Q5: Como a complexidade do contrato inteligente afeta a probabilidade de encontrar bugs?
\end{itemize}
\subsection{Categoria dos protocolos}
\label{sec:org7ab59b5}
\begin{itemize}
\item Derivatives: Protocols for betting with leverage
\item Yield Aggregator: Protocols that aggregated yield from diverse protocols
\item DAO: A decentralized autonomous organization (DAO) is an emerging form of legal structure that has no central governing body and whose members share a common goal to act in the best interest of the entity. Popularized through cryptocurrency enthusiasts and blockchain technology, DAOs are used to make decisions in a bottom-up management approach.
\item Launchpad: Protocols that launch new projects and coins
\item Indexes: Protocols that have a way to track/created the performance of a group of related assets
\item Dexes: Protocols where you can swap/trade cryptocurrency
\item RWA: Protocols that involve Real World Assets, such as house tokenization
\item Algo-Stables: Protocols that provide algorithmic coins to stablecoins
\item CDP: Protocols that mint its own stablecoin using collateralized lending
\end{itemize}
\subsection{Classificação dos bugs}
\label{sec:orge11bb94}
\begin{itemize}
\item O: Out-of-scope
\begin{itemize}
\item We cannot access the source code of the project.
\item Bugs that occur in off-chain components
\item Smart contracts are written in another language
\end{itemize}
\item C01: Mempool Manipulation / Front-Running Vulnerabilities, (e.g sandwich attacks, flash-loan exploits)
\item C02: Reentry attack - Reentrancy vulnerabilities happen when external contract calls are made before internal state updates, allowing an adversary to recursively call back into the contract, exploiting the inconsistent state.
\item C03: Erroneous state updates.
\begin{itemize}
\item C03-1: Missing state update.
\item C03-2: Incorrect state updates, e.g., a state update that should not be there.
\end{itemize}
\item C04: Hardcoded Setting - refers to the practice of embedding fixed values or parameters directly into the source code of a smart contract. This can pose a security risk if the setting needs to be dynamic or adaptable.
\item C05: Privilege escalation and access control issues.
\begin{itemize}
\item C05-1: Privileged functions can be called by anyone or at any time.
\item C05-2: User funds can get locked due to missing/wrong withdraw code
\end{itemize}
\item C06: Wrong Math / Erroneous accounting.
Wrong Math refers to a potential issue where mathematical operations within a smart contract are implemented incorrectly, leading to inaccurate calculations.
\begin{itemize}
\item C06-1: Incorrect calculating order.
\item C06-2: Returning an unexpected value that deviates from the expected semantics specified for the contract.
\item C06-3: Calculations performed with incorrect numbers (e.g., x = a + b ==> x = a + c, incorrect precisions).
\item C06-4: Other accounting errors (e.g., x = a + b ==> x = a - b).
\end{itemize}
\item C07: Broken business logic
Logic vulnerabilities involve flaws in the business logic or protocols of a smart contract, where the implementation matches the developer's intention, but the underlying logic is inherently flawed.
This category includes:
\begin{itemize}
\item C07-1: Unexpected or missing function invocation sequences (e.g., external calls to dependent contracts,  exploitable sequences leading to malicious fund reallocation or manipulation).
\item C07-2: Unexpected environment or contract conditions (e.g., ChainLink returning outdated data or significant slippage occurring).
\item C07-4: Unexpected function arguments.
\end{itemize}
\item C08: Contract implementation-specific bugs. These bugs are difficult to categorize into the above categories.
\item C09: Lack of signature replay protection, e.g missing nonce, hash collision
\item C10: Missing check.
Missing Check refers to a critical oversight in a smart contract's code where a necessary condition or validation is not properly implemented.
\item C11: lack of segregation between users funds
\item C12: Data validation
Data validation vulnerabilities arise when a smart contract does not adequately verify or sanitize inputs, especially those from untrusted sources. This lack of validation can lead to unintended and potentially harmful consequences within the contract’s operations.
\item C13: Whitelist/Blacklist Match
Whitelist/Blacklist Match refers to a potential vulnerability where a smart contract improperly handles addresses based on predefined lists.
\item C14: Arrays
Vulnerabilities related to arrays can arise when developers do not properly handle array indices or fail to validate user inputs.
 would typically be reserved for vulnerabilities that directly arise from mishandling or misinterpreting arrays in the code. For example, if there were out-of-bound reads/writes, deletion mishaps, or issues with array resizing
\item C15: DoS: Denial of Service (DoS) vulnerabilities occur when an attacker can exploit a contract in a way that makes it unresponsive or significantly less efficient. This category includes cases that are not well described by another class and where the primary consequence is contract shut-down or operational inefficiency.
\item C16: Grielf Attack: A gas griefing attack happens when a user sends the amount of gas required to execute the target smart contract, but not its sub calls. In most cases, this results in uncontrolled behavior that could have a dangerous impact on the business logic.
\end{itemize}
\subsection{Dados coletados}
\label{sec:orga96c5ca}
Foi feito a curadoria de 100 bugs classificados com severidade alta
\subsection{Desenvolvimento}
\label{sec:org62cecc7}
\subsection{Categorias}
\label{sec:org8cf685f}
\subsection{Dificuldade}
\label{sec:orgc6cbd80}
\section{Referências}
\label{sec:org3db9a40}
\printbibliography
\end{document}